\documentclass[12pt]{article}

\pagestyle{empty}
\setcounter{secnumdepth}{2}

\topmargin=0cm
\oddsidemargin=0cm
\textheight=22.0cm
\textwidth=16cm
\parindent=0cm
\parskip=0.15cm
\topskip=0truecm
\raggedbottom
\abovedisplayskip=3mm
\belowdisplayskip=3mm
\abovedisplayshortskip=0mm
\belowdisplayshortskip=2mm
\normalbaselineskip=12pt
\normalbaselines

\begin{document}

\vspace*{0.5in}
\centerline{\bf\Large Test Document}

\vspace*{0.5in}
\centerline{\bf\Large Team 3}

\vspace*{0.5in}
\centerline{\bf\Large 27 January 2014}

\vspace*{1.5in}
\begin{table}[htbp]
\caption{Team 3}
\begin{center}
\begin{tabular}{|r | c|}
\hline
Name & ID Number \\
\hline
Mirotchnick, Norman & 4348885 \\
\hline
St-Louis-Fortier, Maxime & 6665810\\
\hline
Rani, Jayanti & 5714583 \\
\hline
Dupuis, Justin & 9690352\\
\hline
Benchetrit, Sidney & 5864585 \\
\hline
Joyejob, Kirti & 6046002 \\
\hline
Crawford, Jordan & 1963031 \\
\hline
Lang, Jonathan & 6298389 \\
\hline
Grosse, Eric & 5983592 \\
\hline
Wei, Chang & 9755829 \\
\hline
\end{tabular}
\end{center}
\end{table}

\clearpage

\section{Introduction}

{\it
The document provides an outline demonstrating Test cases and a few examples of how this is carried out. 
The goal is to show that the methods run correctly with carefully selected test data.
}

\section{Test Plan}

{\it
Each method demanded approximately 5 or more tests per unit.

Since we did unit testing instead of parameterized testing, it took us more time, but got to test all the methods.

}

\subsection{Test Case Unit Testing}
{\it
Validation testing.

We tested to see if each of the unit method is valid
For instance:
}

  @Before

  public static void testEachSetup() {

    // do something before each test

    System.out.println("Prepping Test....");

  }
  
  @After

  public static void testEachCleanup() {

    // do something after each test

    System.out.println("Test Completed!");

  }



\subsection{System Level Test Cases}

{\it Grid Testing
}

The grid has been created and tested through saving.

For example:

  @Test

  public void testSaveTrue() {

	    Grid testGrid = new Grid();

	    testGrid.getCell("A", 1);//should create a cell

	    testGrid.save("test");

	    File file = new File("test.sav");

	    assertTrue(file.exists());

	    file.delete();

  }

{\it Checking the grid if evaluated values are kept
}
For example

public void testLoadSaveEquals2() {
	  
	    Grid testGrid = new Grid();

	    Cell c = testGrid.getCell("A", 1);

	    c.setValue("=2");

	    c.evaluate();
	    


	    testGrid.save("testSave");

	    testGrid = new Grid();

	    testGrid.load("testSave");

	    assertEquals(2.0, testGrid.getCell("A", 1).getEvaluatedValue(),0.00001);

	    File file = new File("testSave.sav");

	    file.delete();

  }


\subsubsection{Test Case 1} \label{tc:1}

\noindent
{\bf Purpose}\\
Selecting a valid cell (i.e Columns A..K, Rows 1..10). The selected cell should lie in the range given.

\noindent
{\bf Input Specification}\\
It restricts the user to select a cell within the appropriate range. Input test data would be any cell in the range such as A2, B5, C4 and so on. 

\noindent
{\bf Expected Output}\\
The system should output the formula or value contained in the cell.

\noindent
{\bf Traces to Use Cases}\\
It allows the user to select a valid cell and in consequence do operations pertaining to the cell like entering a formula or a value or a calculated value due to some formula found in another cell.



\subsubsection{Test Case 2} \label{tc:1}

\noindent
{\bf Purpose}\\
Output spreadsheet as a grid while checking if the data is valid in the cell.

\noindent
{\bf Input Specification}\\
The grid has been saved into memory and loaded back to ensure that it works correctly. Loading and saving in memory have been tested so that the user can retrieve a saved file with his previous work.

\noindent
{\bf Expected Output}\\
The system should load the spreadsheet with all the previous modifications made to it and save it back again for later use. Appropriate error messages will be triggered for any problems encountered for example, selecting a cell out of the given range like Z55.

\noindent
{\bf Traces to Use Cases}\\
Opening, saving the spreadsheet in memory is achieved. The user will be able to quit the application safely without any error messages being triggered.






























\subsection{Subsystem Level Test Cases}

{\it
All test cases for testing at the subsystem level.
}

{\it
One subsection per subsystem
}

\subsubsection{Subsystem X}

\subsection{Unit Test cases}

{\it
All test cases for testing at the unit level.
}

{\it
One subsection per unit
}

\subsubsection{Unit X}

\section{Test Results}

{\it
List the tests, indicating which passed and which did not pass.
List requirements indicating the percentage of tests that passed for that requirement.
}

\section{References}

\appendix

\section{Description of Input Files}

Describe/include test data from input files.

\section{Description of Output Files}

Describe/include test expected output that are output files.

\end{document}
